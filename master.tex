\documentclass{aa}
% \documentclass[referee]{aa}
\usepackage[varg]{txfonts}
\usepackage[separate-uncertainty=true]{siunitx}
\usepackage[version=3]{mhchem}

\sisetup{range-units = brackets}

\def\eps{\varepsilon}
\def\aap{A\&A}
\def\eprint{e-prints}
\def\apj{ApJ}
\def\apjs{ApJS}
\def\apjl{ApJL}
\def\mnras{MNRAS}
\def\aj{AJ}
\def\nat{Nature}
\def\aaps{A\&A Supp.}
\def\prd{Phys. Rev. D}
\def\prl{Phys. Rev. Lett.}
\def\araa{ARA\&A}       % Annual Review of Astron and Astrophys

\begin{document}


\title{High resolution near-IR spectroscopy of Arcturus and 10 Leo}
\subtitle{Refining a near-IR iron line list}


\author{ D.~T.~Andreasen\inst{1,2}
    \and S.~G.~Sousa\inst{1}
    \and E.~Delgado Mena\inst{1}
    \and N.~C.~Santos\inst{1,2}
    \and T.~Lebzelter}


\institute{
Instituto de Astrof\'isica e Ci\^encias do Espa\c{c}o, Universidade do Porto, CAUP, Rua das Estrelas, 4150-762 Porto, Portugal
\email{daniel.andreasen@astro.up.pt}
\and
Departamento de F\'isica e Astronomia, Faculdade de Ci\^encias, Universidade do Porto, Rua Campo Alegre, 4169-007 Porto, Portugal
\and
Departamento de F\'{i}sica, Universidade Federal do Rio Grande do Norte, 59072-970 Natal, RN, Brazil
}





\date{Received ...; accepted ...}

\abstract
% Context
{Effective temperature, surface gravity, and metallicity are basic
spectroscopic stellar parameters necessary to characterize
a star or a planetary system. Reliable atmospheric parameters for
FGK stars have been obtained mostly from methods that relay on high
resolution and high signal-to-noise optical spectroscopy. The
advent of a new generation of high resolution near-IR spectrographs
opens the possibility of using classic spectroscopic methods with
high resolution and high signal-to-noise in the NIR spectral window.}
% Aims
{We aim to compile a new iron line list in the NIR from a solar
spectrum to derive precise stellar atmospheric parameters,
comparable to the ones already obtained from high resolution optical
spectra. The spectral range covers \SI{10000}{\angstrom} to
\SI{25000}{\angstrom}, which is equivalent to the Y, J, H, and K bands.}
% Methods
{Our spectroscopic analysis is based on the iron excitation and ionization
balance done in LTE. We use a high resolution and high signal-to-noise ratio
spectrum of the Sun from the Kitt Peak telescope as a starting point to compile
the iron line list. The oscillator strengths ($\log\mathit{gf}$) of the iron
lines were calibrated for the Sun. The abundance analysis was done using the
MOOG code after measuring equivalent widths of 357 solar iron lines.}
% Results
{We successfully derived stellar atmospheric parameters for the
Sun.
Furthermore, we analysed
HD20010, a F8IV star, from which we derived stellar atmospheric
parameters using the same line list as for the Sun. The spectrum
was obtained from the CRIRES-POP database.
The results are compatible with the ones found in the literature,
confirming the reliability of our line list. However, due to the
quality of the data we obtain large errors.}
% Conclusions
{}



\keywords{data reduction: high resolution spectra --
          stars individual: Arcturus --
          stars individual: HD010853}
\maketitle



\section{Introduction}
\label{sec:introduction}

Effective temperature ($T_\mathrm{eff}$), surface gravity ($\log g$),
and metallicity ([M/H], where iron is normally used as a proxy)
are fundamental atmospheric parameters necessary to characterise a single
star, and to determine other indirect fundamental parameters
such as mass, radius, and age from stellar evolutionary models
\citep[see e.g.][]{Girardi2000,Dotter2008,Baraffe2015}.
Precise and accurate stellar parameters are also essential in
exoplanet searches. Planetary radius and mass are mainly found from
lightcurve analysis and radial velocity analysis, respectively. The
determination of the mass of the planet implies a knowledge of the
stellar mass, while the measurement of the radius of the planet
is dependent on our capability to derive the radius of the star
\citep[see e.g.][]{Torres2008,Ammler2009,Torres2012}.

The derivation of precise stellar atmospheric parameters is not a simple task.
Different approaches often lead to discrepant results
\citep[see e.g.][]{Santos13}. Interferometry is usually considered  an accurate
method for deriving stellar radii \citep[e.g.][]{Boyajian2012}; however, it is
only applicable for bright nearby stars. Asteroseismology, on the other hand,
reveals the inner stellar structure by observing the stellar pulsations at the
surface. From asteroseismology it is possible to measure the surface gravity and
mean density, and therefore to calculate the mass and radius
\citep[e.g.][]{Kjeldsen1995}.

A crucial parameter for the indirect determination of stellar bulk
properties is the effective temperature. In that respect, the infrared
flux method (IRFM) has proven to be reliable for FGK dwarf and
subgiant stars. However, the IRFM needs a priori knowledge of the
bolometric flux, reddening, surface gravity, and stellar metallicity
\citep{Blackwell1977,Ramirez2005b,Casagrande2010}.

Finally, the use of high resolution spectroscopy along with stellar atmospheric
models is an extensively tested method that allows the derivation of the
fundamental parameters of a star \citep[see e.g.][]{Valenti2005,Santos13}. The
procedure depends on the quality of the spectra, their resolution, and
wavelength region. For low resolution spectra ($\lambda/\Delta\lambda <
20\,000$) the preferred method is to fit the overall observed spectrum with a
synthetic one \citep[see e.g.][]{Recio2006}. Higher resolution spectra of slowly
rotating stars (below 10 to 15 \si{km/s})  are in the regime where the
equivalent width (EW) method can be used
\citep[see e.g.][for details]{Andreasen2017}.

The derivation of stellar atmospheric parameters from high resolution spectra in
the optical is now based on a standard procedure
\citep[see e.g.][]{Valenti2005,Sousa2008a}. With the advancement of high
resolution near-infrared (NIR) instruments, we will now be able to use a similar
technique to that used in the optical part of the spectrum
\citep[see e.g.][]{Melendez1999,Sousa2008a,Tsantaki2013,Mucciarelli2013,Bensby2014}.
At the moment, the GIANO spectrograph installed at \emph{Telescopio Nazionale
Galileo} (TNG) is already available \citep{GIANO}, the \emph{infrered doppler
instrument} (IRD) installed at the Subaru telescope \citep{IRD}, as is
\emph{Calar Alto high-Resolution search for M dwarfs with Exoearths with
Near-infrared and optical Échelle Spectrographs} (CARMENES) for the \SI{3.5}{m}
telescope at Calar Alto Observatory \citep{CARMENES}. Two new spectrographs are
planned for the near future: 1) The \emph{CRyogenic InfraRed Echelle
Spectrograph Upgrade Project} (CRIRES+) at the \emph{Very Large Telescope} (VLT)
\citep{CRIRESp} with expected first light in 2017, and 2) \emph{un
SpectroPolarimètre Infra-Rouge A Near-InfraRed Spectropolarimeter} (SPIRou) at
\emph{The Canada-France-Hawaii Telescope} (CFHT) \citep{SPIROU1,SPIROU2} with
expected first light in 2017 as well. The spectral resolutions for these
spectrographs range between $50\,000$ and $100\,000$.

With the advance of NIR spectrographs, we are yet to be to ready for the
analysis of the data arriving at the moment and in the future. The analysis of
stellar spectra is well understood for FGK stars in the optical part of the
spectrum, however some work still needs to be done for the NIR part.

In this work we analyse the atlas of Arcturus (K0III) and the spectrum 10 Leo
(K1III). The atlas of Arcturus was aquired at Kitt Peak National Observatory
using the FTS spectrograph at the Mayall telescope \citep{Hinkle2003} and 10 Leo
from CRIRES \citep{Nicholls2016}. For the analysis we use the iron line list
presented in \citet{Andreasen2016}. This work serve as a continuation of our
previous work.

The paper is organized as follows. In Sect.~\ref{sec:data} we present the data
we have aquired for this work along with some information of the two stars we
will analyse. in Sect.~\ref{sec:refining_the_line_list} we refine the iron line
list in order to get more reliable stellar parameters. The results are presented
in Sect.~\ref{sec:results} before we discuss our results in
Sect.~\ref{sec:conclusion}.




\section{Data}
\label{sec:data}

While the community is currently on the verge to access of a large amount of
high resolution NIR spectra with e.g. the spectrographs used here, the available
spectra at the moment are sparse. We chose to use two stars cooler than the Sun
since we showed in \citet{Andreasen2016} that this method works for a star
hotter than the Sun (HD20010).

We have collected the atlas of Arcturus, one of the brightest stars on the
Nothern hemisphere. Thus it is well studied
\citep[see e.g.][to mention a few]{Griffin1967,McWilliam1990,Ramirez2013}. We
use the atlas from \cite{Hinkle2003} which covers the spectral range of interest
(YJHK bands). Strong telluric features were identified with a spectrum from the
TAPAS web page \citep{Bertaux2014}.

The second spectrum we have achieved is from the CRIRES-POP team
\citep{Nicholls2016}. 10 Leo is a very similar star to Arcturus, which is also
one reason this star was the first to be fully reduced by the team. It is a
great help to be able to compare with the atlas of Arcturus. The main difference
is the metallicity of the two stars, where Arcturus is metal poor and 10 Leo has
solar metallicity. The fully reduced spectrum of 10 Leo is also telluric
corrected using molecfit \citep{Smette2015,Kausch2015}. There are a few gaps in
the spectrum. This is either due to a telluric that could not be properly
removed, low S/R, bad pixels, etc. Rather than giving an uncertain
interpolation, \citet{Nicholls2016} decided to leave small gaps in the data.
This have very little affect on this analysis. However, we were unable to
measure one \ion{Fe}{II} line, which are very important to determine the surface
gravity.

A small summarise of the data is given in Tab.~\ref{tab:data}.




\begin{table*}[htb!]
    \caption{The spectra and spectral type (from Simbad) of our sample with
             the corresponding spectrograph used to acquire the data and its
             spectra resolution. In the last column we show the SNR measured
             with splot in IRAF.}
\label{tab:data}
    \centering
    \begin{tabular}{lrrrr}
      \hline\hline
        Star      & Spectral type & Spectrograph  & Resolution   &  SNR  \\
      \hline
        Arcturus  &      K0III    & FTS           &  $100\;000$  &  300  \\
        10 Leo    &      K1III    & CRIRES        &  $100\;000$  &  300
    \end{tabular}
\end{table*}




\section{Refining the NIR line list}
\label{sec:refining_the_line_list}

Besides testing the line list at cooler effective temperatures with two K stars,
we also want to refine the line list. This includes identifying recurring
outliers, and lines which we are not able to measure, e.g. if a line is amidst a
forrest of telluric lines. Hence, de-blending is nearly impossible.


\section{Results}
\label{sec:results}

We derive the stellar atmospheric parameters in a similar way as described in
\citet{Andreasen2016} using FASMA \citep{Andreasen2017a}.


\subsection{Arcturus}
\label{sec:arcturus}
Arcturus is one of the brightest stars on the night sky with a V magnitude of
-0.05 \citep{Ducati2002}. Hence it has been subject to numerous observations
(add some nice references here...) and is therefore a prime target for testing
the line list by~\cite{Andreasen2016}.

Lines blended with telluric were omitted from the analysis. The equivalent width
(EW) of rest of the lines were measured by hand using the splot function in
IRAF. In the atlas there exist both a summer observation set and a winter
observation set. This is in order to minimize the effect of tellurics at
different spectral regions. As many lines as possible were measured in both
sets, and combined to the final measure line list.

The derivation of the parameters follow exactly the same procedure as used
in~\cite{Andreasen2016}.

\begin{table*}[htb!]
    \caption{The derived parameters for Arcturus with
    fixed surface gravity cut after 3$\sigma$ outlier removal. linelist: arcturus2Cut4ol.moog}
    \label{tab:arcturus}
    \centering
    \begin{tabular}{lllll}
      \hline\hline
                     & $T_\mathrm{eff}$ (K) &  $\log g$ (dex)  &   $\xi_\mathrm{micro}$ (km/s)   & [Fe/H] (dex)      \\
      \hline
        Literature   & $6131 \pm 255$       &  $4.01 \pm 0.60$ &    $1.90 \pm 1.08$              & $-0.23 \pm 0.14$ \\
      \hline
                     & $4363 \pm 75$        &   1.59 (fixed)   &    $1.25 \pm 0.07$              & $-0.34 \pm 0.03$ \\
      \hline
    \end{tabular}
\end{table*}


\subsection{10 Leo}
\label{sec:10Leo}







\section{Conclusion}
\label{sec:conclusion}

Being able to successfully determine parameters for two early K giants, we are
now making the bridge for the line list towards cooler temperatures. The obvious
next step is the even colder M stars. Particular interesting are the M dwarfs
known to be prone forming rocky planets.





\begin{acknowledgements}

This work was supported by Funda\c{c}\~ao para a Ci\^encia e a
Tecnologia (FCT) through the research grants UID/FIS/04434/2013 and
PTDC/FIS-AST/1526/2014. N.C.S., and S.G.S. acknowledge the support from
FCT through Investigador FCT contracts of reference IF/00169/2012, and
IF/00028/2014, respectively, and POPH/FSE (EC) by FEDER funding through
the program “Programa Operacional de Factores de Competitividade
- COMPETE”. E.D.M. and B.J.A. acknowledge the support from FCT in
form of the fellowship SFRH/BPD/76606/2011 and SFRH/BPD/87776/2012,
respectively. This work also benefit from the collaboration of a
cooperation project FCT/CAPES - 2014/2015 (FCT Proc 4.4.1.00 CAPES).

This research has made use of the SIMBAD database operated at CDS,
Strasbourg (France).

This work has made use of the VALD database, operated at Uppsala
University, the Institute of Astronomy RAS in Moscow, and the University
of Vienna.

\end{acknowledgements}


\bibpunct{(}{)}{;}{a}{}{,}
\bibliographystyle{aa}
\bibliography{thesis}


\end{document}
