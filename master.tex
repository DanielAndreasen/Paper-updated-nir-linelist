\documentclass{aa}
% \documentclass[referee]{aa}
\usepackage[varg]{txfonts}
\usepackage[separate-uncertainty=true]{siunitx}
\usepackage[version=3]{mhchem}

\sisetup{range-units = brackets}

\def\eps{\varepsilon}
\def\aap{A\&A}
\def\eprint{e-prints}
\def\apj{ApJ}
\def\apjs{ApJS}
\def\apjl{ApJL}
\def\mnras{MNRAS}
\def\aj{AJ}
\def\nat{Nature}
\def\aaps{A\&A Supp.}
\def\prd{Phys. Rev. D}
\def\prl{Phys. Rev. Lett.}
\def\araa{ARA\&A}       % Annual Review of Astron and Astrophys

\begin{document}


\title{High resolution near-IR spectroscopy of FGKM stars}
\subtitle{Refining a near-IR iron line list}


\author{ D.~T.~Andreasen\inst{1,2}
    \and S.~G.~Sousa\inst{1}
    \and E.~Delgado Mena\inst{1}
    \and N.~C.~Santos\inst{1,2}}


\institute{
Instituto de Astrof\'isica e Ci\^encias do Espa\c{c}o, Universidade do Porto, CAUP, Rua das Estrelas, 4150-762 Porto, Portugal
\email{daniel.andreasen@astro.up.pt}
\and
Departamento de F\'isica e Astronomia, Faculdade de Ci\^encias, Universidade do Porto, Rua Campo Alegre, 4169-007 Porto, Portugal
\and
Departamento de F\'{i}sica, Universidade Federal do Rio Grande do Norte, 59072-970 Natal, RN, Brazil
}





\date{Received ...; accepted ...}

\abstract
% Context
{Effective temperature, surface gravity, and metallicity are basic
spectroscopic stellar parameters necessary to characterize
a star or a planetary system. Reliable atmospheric parameters for
FGK stars have been obtained mostly from methods that relay on high
resolution and high signal-to-noise optical spectroscopy. The
advent of a new generation of high resolution near-IR spectrographs
opens the possibility of using classic spectroscopic methods with
high resolution and high signal-to-noise in the NIR spectral window.}
% Aims
{We aim to compile a new iron line list in the NIR from a solar
spectrum to derive precise stellar atmospheric parameters,
comparable to the ones already obtained from high resolution optical
spectra. The spectral range covers \SI{10000}{\angstrom} to
\SI{25000}{\angstrom}, which is equivalent to the Y, J, H, and K bands.}
% Methods
{Our spectroscopic analysis is based on the iron excitation and
ionization balance done in LTE. We
use a high resolution and high signal-to-noise ratio spectrum of the Sun
from the Kitt Peak telescope as a starting point to compile the iron
line list. The oscillator strengths ($\log\mathit{gf}$) of the iron lines were calibrated for the Sun.
The abundance analysis was done using
the MOOG code after measuring equivalent widths of 357 solar iron lines.}
% Results
{We successfully derived stellar atmospheric parameters for the
Sun.
Furthermore, we analysed
HD20010, a F8IV star, from which we derived stellar atmospheric
parameters using the same line list as for the Sun. The spectrum
was obtained from the CRIRES-POP database.
The results are compatible with the ones found in the literature,
confirming the reliability of our line list. However, due to the
quality of the data we obtain large errors.}
% Conclusions
{}



\keywords{data reduction: high resolution spectra --
          stars individual: Arcturus --
          stars individual: HD010853}
\maketitle



\section{Introduction}
\label{sec:introduction}

Some introduction goes here...

In this work we analyze spectra for a wide range of spectral types
ranging from F to late K using different spectrographs. An overview
of the different spectra and which spectrograph was used to acquire the
data can be seen in Table~\ref{tab:data}.





\section{Data}
\label{sec:data}

\begin{table*}[htb!]
    \caption{The spectra and spectral type (from Simbad) of our sample with
             the corresponding spectrograph used to acquire the data and its
             spectra resolution. In the last column we show the SNR measured
             with splot in IRAF.}
    \label{tab:data}
    \centering
    \begin{tabular}{lrrrr}
      \hline\hline
        Star      & Spectral type & Spectrograph  & Resolution   &  SNR  \\
      \hline
        Arcturus  &      K0III    &  ---          &    ---       &  ---  \\
        HD79210   &      M0V      & CARMENES      &    ---       &  ---  \\
        HD79211   &      M0V      & CARMENES      &    ---       &  ---  \\
        HD285968  &      M2.5V    & CRIRES        &    ---       &  ---  \\
        HD10853   &      K3.5V    & GIANO         &    ---       &  ---  \\
        HD170693  &      K1.5III  & GIANO         &    ---       &  ---  \\
        Gl880     &      M1.5V    & i-Shell       &    ---       &  ---  \\
        Gl250B    &      M2V      & i-Shell       &    ---       &  ---  \\
    \end{tabular}
\end{table*}



\section{Results}
\label{sec:results}



\subsection{Arcturus}
\label{sec:derived_parameters_of_the_sun}
Arcturus is one of the brightest stars on the night sky with a V magnitude of
-0.05 \citep{Ducati2002}. Hence it has been subject to numerous observations
(add some nice references here...) and is therefore a prime target for testing
the line list by \cite{Andreasen2016}.

We use the atlas from \cite{Hinkle2003} which covers the spectral range of
interest. We identify the spectral lines with plot\textunderscore{}fits also presented in
\cite{Andreasen2016}, where we also correct for RV by comparing to to a
synthetic spectrum from the PHOENIX library \citep{Husser2013}. Similarly,
strong telluric features were identified with a spectrum from the TAPAS
web page \citep{Bertaux2014}. Lines blended with telluric were ommited
from the analysis. The EW of rest of the lines were measure by hand using the
splot function in IRAF. In the atlas there exist both a summer observation
set and a winter observation set. This is in order to minimize the effect of
tellurics at different spectral regions. As many lines as possible were
measured in both sets, and combined to the final measure line list.
The results can be seen in Table~\ref{tab:arcturus} and shows excellent
agreement with the literature. Note, that due to the few \ion{Fe}{II} lines
we have to fix the surface gravity. The value was a mean of many literature
values

The derivation of the parameters follow exactly the same procedure as used
in \cite{Andreasen2016}.

\begin{table*}[htb!]
    \caption{The derived parameters for Arcturus with
    fixed surface gravity cut after 3$\sigma$ outlier removal. linelist: arcturus2Cut4ol.moog}
    \label{tab:arcturus}
    \centering
    \begin{tabular}{lllll}
      \hline\hline
                     & $T_\mathrm{eff}$ (K) &  $\log g$ (dex)  &   $\xi_\mathrm{micro}$ (km/s)   & [Fe/H] (dex)      \\
      \hline
        Literature   & $6131 \pm 255$       &  $4.01 \pm 0.60$ &    $1.90 \pm 1.08$              & $-0.23 \pm 0.14$ \\
      \hline
                     & $4363 \pm 75$        &   1.59 (fixed)   &    $1.25 \pm 0.07$              & $-0.34 \pm 0.03$ \\
      \hline
    \end{tabular}
\end{table*}



\subsection{HD010853 (Katie)}
\label{sub:HD010853}
HD010853 is K3 dwarf star with a 8.9 V magnitude \citep{Koen2010}. It has
been subject to several studies (fancy references). We compare our results
with (ref to ELODIE) since they derive the same parameters as us. However,
previous studies agree well. The ELODIE spectrum is public, and we derived
parameters for the optical part as well for this star using the line list
presented in \citet{Sousa2008a}. Our results from the ELODIE spectrum is
similar to (ref to ELODIE) as presented in Table~\ref{tab:hd010853}.

The near-IR spectrum we use of this star was observed with the GIANO
spectrograph installed at Telescopio Nazionale Galileo (TNG).




\begin{table*}[htb!]
    \caption{The derived parameters for HD010853 with
    fixed surface gravity cut after 3$\sigma$ outlier removal. linelist: arcturus2Cut4ol.moog}
    \label{tab:hd010853}
    \centering
    \begin{tabular}{lllll}
      \hline\hline
                     & $T_\mathrm{eff}$ (K) &  $\log g$ (dex)  &   $\xi_\mathrm{micro}$ (km/s)  & [Fe/H] (dex)      \\
      \hline
        Literature   & $6131 \pm 255$       &  $4.01 \pm 0.60$ &    $1.90 \pm 1.08$              & $-0.23 \pm 0.14$ \\
      \hline
                     & $4363 \pm 75$        &   1.59 (fixed)   &    $1.25 \pm 0.07$              & $-0.34 \pm 0.03$ \\
      \hline
    \end{tabular}
\end{table*}




\subsection{Refining the line list}
\label{sub:refining_the_line_list}
Besides testing the line list at cooler effective temperatures
with two K stars, we also want to refine the line list. This
includes identifying recurring outliers, and lines which we are
not able to measure, e.g. if a line is amidst a forrest of
telluric lines. Hence, de-blending is nearly impossible.









\section{Conclusion}
\label{sec:conclusion}
Being able to successfully determine parameters for Arcturus, a K0 giant,
and Katie, a K3 dwarf,
we are now making the bridge for the line list towards cooler temperatures.
The obvious next step is the even colder M stars. Particular interesting
are the M dwarfs known to be prone forming rocky planets.





\begin{acknowledgements}

This work was supported by Funda\c{c}\~ao para a Ci\^encia e a
Tecnologia (FCT) through the research grants UID/FIS/04434/2013 and
PTDC/FIS-AST/1526/2014. N.C.S., and S.G.S. acknowledge the support from
FCT through Investigador FCT contracts of reference IF/00169/2012, and
IF/00028/2014, respectively, and POPH/FSE (EC) by FEDER funding through
the program “Programa Operacional de Factores de Competitividade
- COMPETE”. E.D.M. and B.J.A. acknowledge the support from FCT in
form of the fellowship SFRH/BPD/76606/2011 and SFRH/BPD/87776/2012,
respectively. This work also benefit from the collaboration of a
cooperation project FCT/CAPES - 2014/2015 (FCT Proc 4.4.1.00 CAPES).

This research has made use of the SIMBAD database operated at CDS,
Strasbourg (France).

This work has made use of the VALD database, operated at Uppsala
University, the Institute of Astronomy RAS in Moscow, and the University
of Vienna.

\end{acknowledgements}


\bibpunct{(}{)}{;}{a}{}{,}
\bibliographystyle{aa}
\bibliography{thesis}


\begin{appendix}
\section{An appendix}


\end{appendix}




\end{document}
