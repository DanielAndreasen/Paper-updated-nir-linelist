\documentclass{aa}
% \documentclass[referee]{aa}
\usepackage[varg]{txfonts}
\usepackage[separate-uncertainty=true]{siunitx}
\usepackage[version=3]{mhchem}

\sisetup{range-units = brackets}

\def\eps{\varepsilon}
\def\aap{A\&A}
\def\eprint{e-prints}
\def\apj{ApJ}
\def\apjs{ApJS}
\def\apjl{ApJL}
\def\mnras{MNRAS}
\def\aj{AJ}
\def\nat{Nature}
\def\aaps{A\&A Supp.}
\def\prd{Phys. Rev. D}
\def\prl{Phys. Rev. Lett.}
\def\araa{ARA\&A}

\bibpunct{(}{)}{;}{a}{}{,}


\begin{document}


\title{High resolution near-IR spectroscopy of Arcturus and 10 Leo}
\subtitle{Refining a near-IR iron line list}


\author{ D.~T.~Andreasen\inst{1,2}
    \and S.~G.~Sousa\inst{1}
    \and E.~Delgado Mena\inst{1}
    \and N.~C.~Santos\inst{1,2}
    \and T.~Lebzelter\inst{3}
    \and A.~Mucciarelli\inst{4,5}}


\institute{
  Instituto de Astrof\'isica e Ci\^encias do Espa\c{c}o, Universidade do Porto,
  CAUP, Rua das Estrelas, 4150-762 Porto, Portugal,
  \email{daniel.andreasen@astro.up.pt}
\and
  Departamento de F\'isica e Astronomia, Faculdade de Ci\^encias, Universidade
  do Porto, Rua Campo Alegre, 4169-007 Porto, Portugal
\and
  Institute for Astrophysics, University of Vienna, T\"urkenschanzstrasse 17, 1180
  Vienna, Austria
\and
  Dipartimento di Fisica e Astronomia, Universita' degli Studi di Bologna, Viale
  Berti Pichat, 6/2, 40126, Bologna, Italy
\and
  INAF - Osservatorio Astronomico di Bologna, Via Ranzani 1, 40127, Bologna,
  Italy
}





\date{Received ...; accepted ...}

\abstract
% Context
{Reliable stellar atmospheric parameters for FGK stars have been obtained mostly
from methods that rely on high resolution and high signal-to-noise optical
spectroscopy. The advent of a new generation of high resolution near-IR
spectrographs opens the possibility of using classic spectroscopic methods with
high resolution and high signal-to-noise in the NIR spectral window.}
% Aims
{We aim to obtain precise and accurate atmospheric stellar parameters using
high quality spectra of two early type K giant stars.}
% Methods
{Our spectroscopic analysis is based on the iron excitation and ionization
balance done in LTE and a line list of \ion{Fe}{I} and \ion{Fe}{II} lines in the
NIR domain. The line list is being refined from our previous study, allowing us
to obtain more reliable parameters.}
% Results
{We successfully obtain atmospheric parameters for two K giants in agreement
with average literature values adopted.}
% Conclusions
{With these results we are now extending the line list towards cooler stars,
thus allowing us to explorer the M dwarf stars in the future, known to form
Earth-like planets.}



\keywords{data reduction: high resolution spectra --
          stars individual: Arcturus --
          stars individual: 10 Leo}
\maketitle



\section{Introduction}
\label{sec:introduction}

Effective temperature ($T_\mathrm{eff}$), surface gravity ($\log g$),
and metallicity ([M/H], where iron is normally used as a proxy)
are fundamental atmospheric parameters necessary to characterise a single
star, and to determine other indirectly fundamental parameters
such as mass, radius, and age from stellar evolution models
\citep[see e.g.][]{Girardi2000,Dotter2008,Baraffe2015}.
Precise and accurate stellar parameters are also essential in
exoplanet searches. Planetary radius and mass are mainly found from
transit lightcurve analysis and radial velocity analysis, respectively. The
determination of the mass of the planet implies a knowledge of the
stellar mass, while the measurement of the radius of the planet
is dependent on our capability to derive the radius of the star
\citep[see e.g.][]{Torres2008,Ammler2009,Torres2012}.

The derivation of precise stellar atmospheric parameters is not a simple task.
Different approaches often lead to discrepant results
\citep[see e.g.][]{Torres2010,Lebzelter2012b,Santos13}. Interferometry is
usually considered  an accurate method for deriving stellar radii
\citep[see e.g.][]{Boyajian2012}; however, it is only applicable for bright
nearby stars. Asteroseismology, on the other hand, reveals the inner stellar
structure by observing the stellar pulsations at the surface. From
asteroseismology it is possible to measure the surface gravity and mean density,
and therefore to calculate mass and radius with high precision \citep[see
e.g.][]{Kjeldsen1995}. However, for stars on the main sequence asteroseismic
methods can typically only be applied to FG stars, since the oscillation modes
of K and M dwarfs are likely too weak to be detected even with high precision
spectroscopy or photometry. Moreover, the effective temperature is needed when
applying asteroseismology in order to obtain the surface gravity and the mean
density.

A crucial parameter for the indirect determination of stellar bulk properties is
the effective temperature. In that respect, the infrared flux method (IRFM) has
proven to be reliable for FGK dwarf and subgiant stars. For higher accuracy the
IRFM needs a priori knowledge of the bolometric flux, reddening, surface
gravity, and stellar metallicity
\citep{Blackwell1977,Ramirez2005b,Casagrande2010}.

Finally, the use of high resolution spectroscopy along with stellar atmospheric
models is an extensively tested method that allows the derivation of the
fundamental parameters of a star \citep[see e.g.][]{Valenti2005,Santos13}. The
procedure depends on the quality of the spectra, their resolution, and
wavelength region. A fit to the overall spectrum can be applied for all spectral
resolutions, but are often time consuming \citep[see e.g.][]{Recio2006,Tsantaki2014}.
For resolutions higher than $\lambda/\Delta\lambda < 20\,000$ we can apply the
equivalent width (EW) method \citep[see e.g.][for details]{Tsantaki2013,Andreasen2017a}.
However, while the latter approach is often faster than the synthetic fitting,
it requires higher quality spectra, and the star to be slow rotating (below
$\SI{10}{km/s}$ to $\SI{15}{km/s}$).

Standard procedures are often used to derive stellar atmospheric parameters from
high quality spectra in the optical \citep[see e.g.][]{Valenti2005,Sousa2008a}.
With the advancement of high resolution near-infrared (NIR) instruments, we will
now be able to use a similar technique to that used in the optical part of the
spectrum \citep[see e.g.][]{Melendez1999,Sousa2008a,Tsantaki2013,Mucciarelli2013,Bensby2014}.
At the moment, the GIANO spectrograph installed at \emph{Telescopio Nazionale
Galileo} (TNG) is already available \citep{GIANO}, as is the \emph{infrared
Doppler instrument} (IRD) installed at the Subaru telescope \citep{IRD},
\emph{Calar Alto high-Resolution search for M dwarfs with Exoearths with
Near-infrared and optical Échelle Spectrographs} (CARMENES) for the \SI{3.5}{m}
telescope at Calar Alto Observatory \citep{CARMENES}, and iShell at the
\emph{InfraRed Telescope Facility} \citep{ishell1,ishell2}. Three new
spectrographs are planned for the near future: 1) The \emph{CRyogenic InfraRed
Echelle Spectrograph Upgrade Project} (CRIRES+) at the \emph{Very Large
Telescope} (VLT) \citep{CRIRESp} with expected first light in 2017, 2) \emph{un
SpectroPolarimètre Infra-Rouge A Near-InfraRed Spectropolarimeter} (SPIRou) at
\emph{The Canada-France-Hawaii Telescope} (CFHT) \citep{SPIROU1,SPIROU2} with
expected first light in 2017 as well, and 3) NIRPS at the ESO 3.6m telescope in
La Silla \citep{NIRPS}. The spectral resolutions for these spectrographs range
between $50\,000$ and $100\,000$.

With the advance of the next generation NIR spectrographs, we are still
preparing the data analysis of stellar spectra, in particular how to get
reliable atmospheric parameters \citep[see e.g.][]{Onehag2012,Lindgren2016,Andreasen2016}.
The analysis of stellar spectra is well understood for FGK stars in the optical
part of the spectrum, however some work still needs to be done for the NIR part.

We continue our series of studies to explore the use of the NIR domain to derive
stellar parameters for FGK and M stars. In particular, here we analyse the atlas
of Arcturus and the spectrum of 10 Leo. For the analysis we use the iron line
list presented in \citet{Andreasen2016} (referred to as Paper I). In Paper I we
successfully tested our method on a slightly hotter star than the Sun, while in
this work we aim to test the method on cooler stars. The strength of the NIR
domain over the optical becomes clear when we move towards the cooler stars.
Here we see less continuum depression and line blending due to in particular
molecular features. Moreover, the cooler stars emit more light in the NIR domain
than the optical, and with the lightest stars being intrinsically faint, we thus
obtain the majority of the flux here.



\section{Data}
\label{sec:data}

While the community is currently on the verge to access large amount of high
resolution NIR spectra, the available spectra at the moment are sparse. We chose
to use two stars cooler than the Sun since we used a hotter star (HD 20010) than
the Sun in Paper I. The method used in Paper I and here is determining the iron
abundances on a number of lines from their measured EWs. Then we impose
ionization balance between \ion{Fe}{I} and \ion{Fe}{II} lines, and excitation
balance for all \ion{Fe}{I} lines, by changing the atmospheric parameters for
the model atmosphere \citep[][is used here]{Kurucz1993}.

We have used the atlas of Arcturus (acquired at Kitt Peak National Observatory
using the FTS spectrograph at the Mayall telescope), one of the brightest stars
on the Northern hemisphere. Thus it is well studied \citep[see e.g.][to mention
just a few]{Griffin1967,McWilliam1990,Ramirez2013}. We use the atlas from
\cite{Hinkle2003} which covers the spectral range of interest (YJHK bands).
Strong telluric features were identified with a spectrum from the TAPAS web page
\citep{Bertaux2014}. The atlas also comes with a telluric standard and the ratio
of the two spectra in order to correct for the tellurics. The telluric spectrum
from TAPAS is only used for telluric line identification. We use both the
telluric corrected and non-corrected spectrum.

The spectrum for 10 Leo is from the CRIRES-POP team \citep{Nicholls2016}. 10 Leo
is very similar to Arcturus, which is also one reason this star was the first to
be fully reduced by the CRIRES-POP team. The spectrum is divided into each band
YJ (only together), H, K, L, and M. We use only the first three. Some small gaps
are present in the spectrum due to tellurics that could not be properly removed,
low S/N, bad pixels, etc. Rather than giving an uncertain interpolation,
\citet{Nicholls2016} decided to leave small gaps in the data. This has very
little effect on our line by line analysis. However, we were unable to measure
one \ion{Fe}{II} line due to the gaps, which are generally important to
determine the surface gravity.

The data for the two stars are very similar in terms of S/N (around 300 as
measured by IRAF in a continuum region in the YJ band), resolution
(approximately $100\;000$), and spectral coverage. In Fig.~\ref{fig:both} we
compare the spectra of the two stars in a region with some of the iron lines
used for the analysis described below.

\begin{figure*}[htpb!]
    \centering
    \includegraphics[width=1.0\linewidth]{figures/bothspectra.pdf}
    \caption{Sample spectra of the two stars, in blue is Arcturus, and green is
             10 Leo with an 0.15 offset. We mark the location of \ion{Fe}{I}
             lines in the region.}
    \label{fig:both}
\end{figure*}





\section{Refining the NIR line list}
\label{sec:refining_the_line_list}

In Paper I we prepared a \ion{Fe}{I} and \ion{Fe}{II} line list in the NIR
domain. This line list was calibrated using a solar spectrum, and successfully
used to derive atmospheric parameters for a late F star. Here we will go one
step forward, and test this line list for early K type stars. Besides testing
the line list from Paper I at cooler effective temperatures with two K stars, it
is a primary goal of this work to refine the line list. This includes
identifying recurring outliers (both from the work done in Paper I and in this
work), and lines which we are not able to measure, e.g. if a line is amidst a
forest of telluric lines. To identify these lines the solar atlas used in Paper
I was revisited. In total 211/295 \ion{Fe}{I} lines and 8/13 \ion{Fe}{II} lines
were removed in the process. Most of these were blended lines with either
tellurics or other stellar lines. This procedure leaves us with 84 \ion{Fe}{I}
lines and 5 \ion{Fe}{II} lines. These lines should be the best for deploying our
technique of determining atmospheric stellar parameters.

During a second look at the Solar spectrum, the EW of the lines were measured by
hand (this had previously been done automatically with ARES). Since we
re-measured the EWs, the $\log \mathrm{gf}$ values had to be re-calibrated
again. Here we simply change the $\log \mathrm{gf}$ values for the measured EW
until the abundance of a given line is equal to that of the Sun, using the same
solar atmosphere model as in Paper I. The mean change in $\log \mathrm{gf}$ for
common lines is $-0.09 \pm 0.16$. The line list with the updated $\log
\mathrm{gf}$ is presented in Appendix~\ref{app:linelist}.

The \ion{Fe}{II} lines are used to determine $\log g$ by imposing ionization
balance with the average \ion{Fe}{I} abundance. However, the low number of
\ion{Fe}{II} lines available is a concern, since the average abundance of
\ion{Fe}{II} is affected more by outliers compared to the numerous \ion{Fe}{I}
lines.



\section{Results}
\label{sec:results}

We derive the stellar atmospheric parameters in the same way as described in
Paper I using the new minimization tool, FASMA \citep{Andreasen2017a}. We use
ATLAS9 atmosphere models during the derivation \citep{Kurucz1993}. The EWs are
measured for both stars automatically with ARES \citep{Sousa2015a} and by hand
with splot in IRAF. We compare the derived stellar parameters from the two
measured sets of EWs, and with average adopted literature values.


\subsection{Revisiting HD 20010}
\label{sec:hd20010}

As a first step we revisit HD 20010 for which we derived atmospheric stellar
parameters in Paper I using the newly revised line list presented in this paper.
The results are presented in Tab.~\ref{tab:hd20010} along with the average
literature values (see Paper I and references therein). We see better agreement
with the average literature values adopted (especially $[\ion{Fe}/\ion{H}]$ and
$\log g$), and smaller errors with the updated results. This suggest that the
new line list is more reliable.

\begin{table*}[htb!]
    \caption{Updated results for HD 20010 using the shorter line list and new
             oscillator strengths.}
    \label{tab:hd20010}
    \centering
    \begin{tabular}{lllll}
      \hline\hline
                      & $T_\mathrm{eff}$ (K) &  $\log g$ (dex)  &   $\xi_\mathrm{micro}$ (km/s)   & [Fe/H] (dex)     \\
      \hline
        Literature    & $6131 \pm 255$       &  $4.01 \pm 0.60$ &    $1.90 \pm 1.08$              & $-0.23 \pm 0.14$ \\
      \hline
        This work     & $6157 \pm 180$       &  $4.06 \pm 0.76$ &    $1.62 \pm 0.44$              & $-0.18 \pm 0.11$ \\
        This work     & $6153 \pm 176$       &   4.01 (fixed)   &    $1.68 \pm 0.40$              & $-0.18 \pm 0.11$ \\
      \hline
        Paper I       & $6116 \pm 224$       &  $4.21 \pm 0.58$ &    $2.45 \pm 0.45$              & $-0.14 \pm 0.14$ \\
        Paper I       & $6144 \pm 212$       &   4.01 (fixed)   &    $2.66 \pm 0.42$              & $-0.13 \pm 0.29$ \\
      \hline
    \end{tabular}
\end{table*}




\subsection{Arcturus}
\label{sec:arcturus}

Arcturus is one of the brightest stars on the night sky with a V magnitude of
-0.05 \citep{Ducati2002}. Hence it is a prime target for testing the updated
line list with numerous measurements of the atmospheric parameters as mentioned
above.

The atlas consists of both a summer observation set and a winter observation
set. This is in order to minimize the effect of tellurics at different spectral
regions. A comparison between the two sets of measured EWs - both the manual
measurements using IRAF and the automatic measurements using ARES - are shown in
Fig.~\ref{fig:EWcomp}. The automatic EW measurements for the summer set and
winter set shows excellent agreement. This means that the two data sets are very
similar, thus we chose to only manually measure the EWs for one set (summer). We
did, however, measure a few lines from the winter data set to verify the
agreement. For both the automatically and manually measured EWs, we discard all
lines with an EW below $\SI{5}{m}$\AA{} and above $\SI{150}{m}$\AA{} before
continuing the analysis. Lines outside this range are either too weak to be
reliable measured or are so strong that the we are not able to fit a Gaussian to
the profile. Especially the wings of the absorption lines are a problem for
strong lines. For all three sets of measured EWs (summer and winter observations
automatically, and summer manually), parameters were derived with and without
$\log g$ set to a fixed value (1.69\,dex, the average literature value adopted).
The derivation of the parameters follow the procedure presented in Paper I. We
use the minimization routine from \citet{Andreasen2017a}. After we reach
convergence using all the iron lines we were able to measure, one outlier above
$3\sigma$ in abundance were removed, and the minimization routine was restarted.
This process was done iteratively until there were no more outliers. The final
results are presented in Tab.~\ref{tab:arcturus} together with mean parameters
from the literature.


\begin{figure}[htpb!]
    \centering
    \includegraphics[width=1.0\linewidth]{figures/EWcomp.pdf}
    \caption{Top figure: Difference of the automatic EW measurements between the
             summer observations and winter observations from the Arcturus
             spectra. Bottom figure: Same as above, but with manual measurements
             from ARES (summer) and automatic measurements (summer).}
    \label{fig:EWcomp}
\end{figure}


\begin{table*}[htb!]
    \caption{The derived parameters for Arcturus with and without fixed surface
             gravity. The literature values are a simple mean of all the
             available parameters on Simbad with the corresponding standard
             error. There is no microturbulence available, so we derived it
             using the empirical relation from \citet{Adibekyan2015} for each
             set of parameters.}
    \label{tab:arcturus}
    \centering
    \begin{tabular}{lllll}
      \hline\hline
                      & $T_\mathrm{eff}$ (K) &  $\log g$ (dex)  &   $\xi_\mathrm{micro}$ (km/s)   & [Fe/H] (dex)     \\
      \hline
        Literature    & $4306 \pm 100$       &  $1.69 \pm 0.32$ &    $1.92 \pm 0.15$              & $-0.54 \pm 0.11$ \\
      \hline
        IRAF          & $4380 \pm  79$       &  $0.64 \pm 0.33$ &    $1.14 \pm 0.09$              & $-0.49 \pm 0.07$ \\
        IRAF          & $4212 \pm  77$       &   1.69 (fixed)   &    $1.25 \pm 0.08$              & $-0.37 \pm 0.03$ \\
      \hline
        ARES (summer) & $4439 \pm  63$       &  $1.20 \pm 0.20$ &    $1.55 \pm 0.10$              & $-0.58 \pm 0.06$ \\
        ARES (summer) & $4348 \pm  75$       &   1.69 (fixed)   &    $1.58 \pm 0.09$              & $-0.53 \pm 0.03$ \\
        ARES (winter) & $4436 \pm  67$       &  $0.55 \pm 1.77$ &    $1.35 \pm 0.09$              & $-0.56 \pm 0.07$ \\
        ARES (winter) & $4233 \pm 109$       &   1.69 (fixed)   &    $1.43 \pm 0.09$              & $-0.49 \pm 0.04$ \\
      \hline
        Weighted mean & $4421 \pm  40$       &  $0.96 \pm 0.60$ &    $1.34 \pm 0.05$              & $-0.55 \pm 0.04$ \\
        Weighted mean & $4269 \pm  51$       &   1.69 (fixed)   &    $1.41 \pm 0.05$              & $-0.46 \pm 0.02$ \\
      \hline
    \end{tabular}
\end{table*}

We generally see good agreement between the derived parameters and the average
values from the literature adopted. The only parameter being difficult to
measure is the surface gravity due to the low number of \ion{Fe}{II} lines in
the NIR. It is very important to derive the metallicity accurately, and we
report good results overall, but especially with the automatic measurements,
compared to literature values. When measuring the EWs by hand, we might have
systematically overestimated the continuum, resulting in higher
$[\ion{Fe}/\ion{H}]$. For visualization the parameters are plotted (except
$\xi_\mathrm{micro}$) in Fig.~\ref{fig:arcturus}. Here the histogram shows the
literature values collected from Simbad while the vertical black line is our
final value with gray shaded errorbar. The plot shows that our values are in
very good agreement with the literature results, supporting the quality and
reliability of our analysis.

\begin{figure}[htpb!]
    \centering
    \includegraphics[width=1.0\linewidth]{figures/ArcturusParams.pdf}
    \caption{Histogram of the different sets of literature parameters of
             Arcturus (except $\xi_\mathrm{micro}$). The black vertical line are
             our derived parameters, and the gray shaded area are the errors on
             the corresponding parameters.}
    \label{fig:arcturus}
\end{figure}



\subsection{10 Leo}
\label{sec:10Leo}

The approach for determining the atmospheric stellar parameters for 10 Leo is
identical to Arcturus. We use ARES on each band (YJ, H, and K-band) separately.
For the small gaps in the spectrum, we simply set the flux to 1, since the
spectrum is already normalized. This will also prevent ARES to identify and
measure any lines in these regions. The EWs from the three regions are combined
to one final line list used for the determination of the parameters. The EWs
were also measured by hand using IRAF. We list the results in
Tab.~\ref{tab:10Leo} alongside with an average of literature values taken from
Simbad. The final results and five collected literature values are presented in
Fig.~\ref{fig:10leo}.

\begin{table*}[htb!]
    \caption{Results from 10 Leo presented in the same way as for
             Tab.~\ref{tab:arcturus}.}
    \label{tab:10Leo}
    \centering
    \begin{tabular}{lllll}
      \hline\hline
                      & $T_\mathrm{eff}$ (K) &  $\log g$ (dex)  &   $\xi_\mathrm{micro}$ (km/s)   & [Fe/H] (dex)     \\
      \hline
        Literature    & $4720 \pm  42$       &  $2.54 \pm 0.11$ &    $1.59 \pm 0.02$              & $ 0.00 \pm 0.03$ \\
      \hline
        IRAF          & $4835 \pm  85$       &  $2.41 \pm 0.41$ &    $1.28 \pm 0.08$              & $ 0.09 \pm 0.06$ \\
        IRAF          & $4768 \pm  88$       &   2.54 (fixed)   &    $1.20 \pm 0.08$              & $ 0.01 \pm 0.05$ \\
      \hline
        ARES          & $4805 \pm  98$       &  $2.42 \pm 0.61$ &    $1.23 \pm 0.10$              & $-0.01 \pm 0.07$ \\
        ARES          & $4768 \pm 105$       &   2.54 (fixed)   &    $1.20 \pm 0.10$              & $-0.01 \pm 0.06$ \\
      \hline
        Weighted mean & $4821 \pm  65$       &  $2.41 \pm 0.37$ &    $1.26 \pm 0.06$              & $ 0.04 \pm 0.05$ \\
        Weighted mean & $4768 \pm  69$       &   2.54 (fixed)   &    $1.20 \pm 0.06$              & $ 0.05 \pm 0.04$ \\
      \hline
    \end{tabular}
\end{table*}

Generally the derived parameters are in excellent agreement with the literature
values listed here. We were able to derive good $\log g$ values, although with
larger errors compared to the results from the literature.

\begin{figure}[htpb!!]
    \centering
    \includegraphics[width=1.0\linewidth]{figures/10LeoParams.pdf}
    \caption{Literature values (blue) and the two results from this work (green)
             with and without $\log g$ fixed. The errorbars on the literature
             values are either those presented in the corresponding paper, or in
             the cases none were presented we give an error of $\SI{100}{K}$ for
             $T_\mathrm{eff}$, 0.10\,dex for $\log g$, and 0.05\,dex for
             $[\ion{Fe}/\ion{H}]$. The horizontal lines are the average
             literature values adopted.
             References:
             Ref [1]: \citet{Luck2015},
             Ref [2]: \citet{Park2013},
             Ref [3]: \citet{Massarotti2008},
             Ref [4]: \citet{Soubiran2008}, and
             Ref [5]: \cite{daSilva2011}.}
    \label{fig:10leo}
\end{figure}



\section{Discussion}
\label{sec:discussion}

\subsection{The role of $\log g$}

One of the most difficult atmospheric stellar parameters to get from a spectrum
is the surface gravity. For this we need the lines of pressure sensitive ionized
atoms such as \ion{Fe}{II}. However, they are more sparse than neutral iron,
\ion{Fe}{I}, making the determination more challenging. This is true in the
optical \citep[see e.g. the discussion by][]{Mortier2013c}, and even more in the
NIR (see e.g. Paper I). One solution to this problem is to fix the value of
surface gravity and derive the other parameters. With the parallaxes from Gaia
\citep{GAIA} we will have access to accurate $\log g$. However, this requires a
prior knowledge of the mass from e.g. isochrones, and $T_\mathrm{eff}$. By
iteratively obtaining the $T_\mathrm{eff}$ from spectroscopy and the
corresponding $\log g$ from the parallaxes, we can obtain reliable
$T_\mathrm{eff}$, $\log g$, and $[\ion{Fe}/\ion{H}]$. Without the final
parallaxes from Gaia we may yet only rely on literature values for $\log g$. As
seen from Fig.~\ref{fig:arcturus}, the distribution of $\log g$ values from the
literature are rather disperse. Since there is a dependence between the other
derived parameters with $\log g$, simply using a mean value as a reference value
can lead to misleading parameters. To verify the impact of using the wrong $\log
g$ as baseline, we tested what was the $T_\mathrm{eff}$ and $[\ion{Fe}/\ion{H}]$
that we derive by setting $\log g$ fixed to values between 0.9\,dex and
2.2\,dex, i.e., in the range of the literature values found. The results show
that $T_\mathrm{eff}$ and $[\ion{Fe}/\ion{H}]$ can change by $\SI{200}{K}$ and
0.21\,dex, respectively. This is most likely the origin of the small
discrepancies seen for the parameters of Arcturus when the $\log g$ is fixed and
free.

Note that the ionized iron lines are not only sparse, they are also rather weak.
The lowest measured EW for an \ion{Fe}{II} line is $\SI{7.8}{m}$\AA{} (in
Arcturus), while the highest measured value is $\SI{20.7}{m}$\AA{} (in 10 Leo).
However, with the upcoming high quality spectra for the NIR, the community
should still be able to measure these \ion{Fe}{II} lines.


\subsection{Proper data reduction}

The relative novelty of NIR high resolution spectroscopy is reflected on a
number of problems regarding the available spectra that made our analysis
particularly difficult. For instance, in Paper I we had to deal with a less
reliable wavelength calibration for the spectrum of HD 20010. This meant the
wavelength was stretched when compared to a synthetic spectrum, which is
discussed in more detail by \citet{Nicholls2016}. The poor wavelength
calibration for HD 20010 most likely caused bad EW measurements. In addition,
the spectrum was not corrected for telluric lines which also caused minor
deviation from the true EW when measured. Another reason was the non-refined
line list used, which we have attempted to correct for here. The refined line
list has made the derivation of the metallicity more reliable compared with the
adopted literature as it is demonstrated in Sec.~\ref{sec:hd20010}. It is
expected that even better results will be obtained for this star once the final
spectrum is presented by the CRIRES-POP team.

All the above problems we had with HD 20010 have been solved for 10 Leo, and it
is clear the results are of much higher quality. This can be seen by the smaller
errors we have on our parameters, and the good agreement of all parameters
compared with the literature. Therefore, it may be needed that a telluric
correction is applied to the spectrum before atmospheric stellar parameters can
be determined reliably. However, with our limited sample it is hard to make a
clear conclusion yet.



\section{Conclusion}
\label{sec:conclusion}

In this paper we present a refined \ion{Fe}{I} and \ion{Fe}{II} line list in the
near-IR domain. This line list has been used to derive new parameters for the
late F-star HD 20010, as well as for two K-giants (Arcturus and 10 Leo). The
results show that the stellar atmospheric parameters derived using our line list
are perfectly compatible with the literature values. We are thus now extending
the line list towards cooler temperatures. With the updated results for HD
20010, and the results for Arcturus and 10 Leo, we are now reaching the same
precision that has been reached in the optical for similar spectral types using
the same methodology. The obvious next step is to approach the even cooler M
stars. Particular interesting are the M dwarf stars, known to be prone forming
rocky planets. As important as cooler stars, we have yet to test our line list
on any dwarf stars other than the Sun for which our line list is calibrated. The
upcoming spectral library from CARMENES (priv. comm. with P. Amado) will provide
the community with high quality spectra and allow us to extend our test to many
different spectral types of interest.





\begin{acknowledgements}

We thank Jos\'e Caballero for many useful for comments during the process which
led to this paper. He has been most kind providing help whenever needed.

This work was supported by Funda\c{c}\~ao para a Ci\^encia e a Tecnologia, FCT,
(ref. UID/FIS/04434/2013, PTDC/FIS-AST/1526/2014, and PTDC/FIS-AST/7073/2014)
through national funds and by FEDER through COMPETE2020 (ref.
POCI-01-0145-FEDER-007672, POCI-01-0145-FEDER-016886, and
POCI-01-0145-FEDER-016880). N.C.S., and S.G.S. acknowledge the support from FCT
through Investigador FCT contracts of reference IF/00169/2012, and
IF/00028/2014, respectively, and POPH/FSE (EC) by FEDER funding through the
program “Programa Operacional de Factores de Competitividade - COMPETE”. E.D.M
acknowledge the support from the FCT in the form of the grants
SFRH/BPD/76606/2011.

This research has made use of the SIMBAD database operated at CDS, Strasbourg
(France).

\end{acknowledgements}


\bibliographystyle{aa}
\bibliography{thesis}

\begin{appendix}

\section{Complete refined line list}
\label{app:linelist}
The complete refined line list with Solar EWs measured by hand using IRAF.

\begin{onecolumn}
  \begin{longtable}{cclrr}
      \caption{\label{tab:linelist} Refined line list with all \ion{Fe}{I} and
               \ion{Fe}{II} lines and corresponding atomic data, including the
               updated oscillator strengths. This table is available online.}\\
        \hline\hline
          Wavelength (\AA) & Element        & EP                   (eV)  &  $\log \mathrm{gf}$  &  Solar EW (m\AA)    \\
        \hline
        \endfirsthead
        \caption{continued.}\\
        \hline\hline
          Wavelength (\AA) & Element        & EP                   (eV)  &  $\log \mathrm{gf}$  &  Solar EW (m\AA)    \\
        \hline
        \endhead
          10065.05         &  \ion{Fe}{I}   &           4.83             &        -0.279        &     94.0            \\
          10080.42         &  \ion{Fe}{I}   &           5.10             &        -1.964        &      5.9            \\
          10081.39         &  \ion{Fe}{I}   &           2.42             &        -4.512        &      6.9            \\
          10086.24         &  \ion{Fe}{I}   &           2.95             &        -3.978        &      7.0            \\
          10137.10         &  \ion{Fe}{I}   &           5.09             &        -1.736        &      9.8            \\
          10142.84         &  \ion{Fe}{I}   &           5.06             &        -1.554        &     14.9            \\
          10145.56         &  \ion{Fe}{I}   &           4.80             &        -0.118        &    109.0            \\
          10155.16         &  \ion{Fe}{I}   &           2.18             &        -4.336        &     16.2            \\
          10156.51         &  \ion{Fe}{I}   &           4.59             &        -2.109        &     12.2            \\
          10167.47         &  \ion{Fe}{I}   &           2.20             &        -2.319        &    125.7            \\
          10195.11         &  \ion{Fe}{I}   &           2.73             &        -3.608        &     22.6            \\
          10216.31         &  \ion{Fe}{I}   &           4.73             &         0.047        &    129.9            \\
          10218.41         &  \ion{Fe}{I}   &           3.07             &        -2.893        &     40.9            \\
          10265.22         &  \ion{Fe}{I}   &           2.22             &        -4.648        &      8.1            \\
          10307.45         &  \ion{Fe}{I}   &           4.59             &        -2.432        &      6.4            \\
          10332.33         &  \ion{Fe}{I}   &           3.63             &        -3.131        &     10.5            \\
          10340.89         &  \ion{Fe}{I}   &           2.20             &        -3.665        &     46.6            \\
          10347.97         &  \ion{Fe}{I}   &           5.39             &        -0.717        &     37.0            \\
          10353.81         &  \ion{Fe}{I}   &           5.39             &        -0.989        &     24.2            \\
          10364.06         &  \ion{Fe}{I}   &           5.45             &        -1.100        &     18.0            \\
          10379.00         &  \ion{Fe}{I}   &           2.22             &        -4.236        &     18.7            \\
          10388.75         &  \ion{Fe}{I}   &           5.45             &        -1.471        &      8.7            \\
          10395.80         &  \ion{Fe}{I}   &           2.18             &        -3.435        &     61.3            \\
          10423.03         &  \ion{Fe}{I}   &           2.69             &        -3.658        &     22.9            \\
          10423.74         &  \ion{Fe}{I}   &           3.07             &        -3.119        &     29.9            \\
          10469.65         &  \ion{Fe}{I}   &           3.88             &        -1.277        &     89.3            \\
          10532.24         &  \ion{Fe}{I}   &           3.93             &        -1.650        &     64.4            \\
          10555.65         &  \ion{Fe}{I}   &           5.45             &        -1.282        &     13.1            \\
          10577.14         &  \ion{Fe}{I}   &           3.30             &        -3.222        &     17.2            \\
          10616.72         &  \ion{Fe}{I}   &           3.27             &        -3.306        &     15.6            \\
          10725.19         &  \ion{Fe}{I}   &           3.64             &        -2.948        &     15.7            \\
          10753.00         &  \ion{Fe}{I}   &           3.96             &        -2.077        &     39.7            \\
          10780.69         &  \ion{Fe}{I}   &           3.24             &        -3.553        &     10.4            \\
          10783.05         &  \ion{Fe}{I}   &           3.11             &        -2.786        &     47.0            \\
          10818.28         &  \ion{Fe}{I}   &           3.96             &        -2.160        &     35.6            \\
          10863.52         &  \ion{Fe}{I}   &           4.73             &        -0.877        &     67.1            \\
          10884.26         &  \ion{Fe}{I}   &           3.93             &        -2.129        &     39.1            \\
          10896.30         &  \ion{Fe}{I}   &           3.07             &        -2.911        &     42.9            \\
          11013.24         &  \ion{Fe}{I}   &           4.80             &        -1.240        &     42.4            \\
          11026.79         &  \ion{Fe}{I}   &           3.94             &        -2.517        &     21.2            \\
          11119.80         &  \ion{Fe}{I}   &           2.85             &        -2.452        &     84.8            \\
          11641.80         &  \ion{Fe}{I}   &           4.58             &        -2.116        &     15.6            \\
          11778.42         &  \ion{Fe}{I}   &           5.34             &        -1.708        &      8.4            \\
          12053.08         &  \ion{Fe}{I}   &           4.56             &        -1.602        &     41.3            \\
          12119.50         &  \ion{Fe}{I}   &           4.59             &        -1.897        &     25.0            \\
          12213.34         &  \ion{Fe}{I}   &           4.64             &        -2.006        &     19.1            \\
          12227.11         &  \ion{Fe}{I}   &           4.61             &        -1.408        &     51.5            \\
          12244.92         &  \ion{Fe}{I}   &           3.64             &        -3.222        &     11.8            \\
          12340.48         &  \ion{Fe}{I}   &           2.28             &        -4.680        &      9.4            \\
          12342.92         &  \ion{Fe}{I}   &           4.64             &        -1.545        &     42.1            \\
          12510.52         &  \ion{Fe}{I}   &           4.96             &        -1.930        &     12.9            \\
          12557.00         &  \ion{Fe}{I}   &           2.28             &        -4.026        &     33.8            \\
          12615.93         &  \ion{Fe}{I}   &           4.64             &        -1.686        &     35.7            \\
          12638.70         &  \ion{Fe}{I}   &           4.56             &        -0.679        &    112.3            \\
          12807.15         &  \ion{Fe}{I}   &           3.64             &        -2.649        &     37.1            \\
          12808.24         &  \ion{Fe}{I}   &           4.99             &        -1.811        &     16.4            \\
          12824.86         &  \ion{Fe}{I}   &           3.02             &        -3.612        &     20.1            \\
          12840.57         &  \ion{Fe}{I}   &           4.96             &        -1.612        &     25.3            \\
          12879.77         &  \ion{Fe}{I}   &           2.28             &        -3.525        &     68.7            \\
          12896.12         &  \ion{Fe}{I}   &           4.91             &        -1.713        &     23.2            \\
          12933.01         &  \ion{Fe}{I}   &           5.02             &        -1.879        &     13.9            \\
          12934.67         &  \ion{Fe}{I}   &           5.39             &        -1.103        &     30.9            \\
          13014.84         &  \ion{Fe}{I}   &           5.45             &        -1.542        &     12.3            \\
          13352.17         &  \ion{Fe}{I}   &           5.31             &        -0.355        &     94.4            \\
          13392.10         &  \ion{Fe}{I}   &           5.35             &        -0.105        &    115.1            \\
          15194.49         &  \ion{Fe}{I}   &           2.22             &        -4.808        &     14.1            \\
          15201.57         &  \ion{Fe}{I}   &           5.49             &        -1.315        &     29.0            \\
          15207.53         &  \ion{Fe}{I}   &           5.38             &         0.311        &    215.9            \\
          15335.38         &  \ion{Fe}{I}   &           5.41             &         0.252        &    205.2            \\
          15490.34         &  \ion{Fe}{I}   &           2.20             &        -4.787        &     16.1            \\
          15593.74         &  \ion{Fe}{I}   &           5.03             &        -1.796        &     28.0            \\
          15611.15         &  \ion{Fe}{I}   &           3.42             &        -2.966        &     51.6            \\
          15631.95         &  \ion{Fe}{I}   &           5.35             &         0.171        &    207.0            \\
          15648.51         &  \ion{Fe}{I}   &           5.43             &        -0.633        &     93.8            \\
          15676.58         &  \ion{Fe}{I}   &           5.11             &        -1.848        &     22.3            \\
          16198.50         &  \ion{Fe}{I}   &           5.41             &        -0.376        &    131.4            \\
          17420.83         &  \ion{Fe}{I}   &           3.88             &        -3.628        &      6.7            \\
          19923.34         &  \ion{Fe}{I}   &           5.02             &        -1.536        &     49.7            \\
          21851.38         &  \ion{Fe}{I}   &           3.64             &        -3.578        &     12.7            \\
          22257.11         &  \ion{Fe}{I}   &           5.06             &        -0.704        &    132.5            \\
          22380.80         &  \ion{Fe}{I}   &           5.03             &        -0.377        &    179.4            \\
          22392.88         &  \ion{Fe}{I}   &           5.10             &        -1.330        &     60.8            \\
          22619.84         &  \ion{Fe}{I}   &           4.99             &        -0.564        &    158.2            \\
          23308.48         &  \ion{Fe}{I}   &           4.08             &        -2.705        &     31.3            \\
          10427.31         &  \ion{Fe}{II}  &           6.08             &        -1.575        &     13.7            \\
          10501.50         &  \ion{Fe}{II}  &           5.55             &        -1.861        &     19.5            \\
          10862.64         &  \ion{Fe}{II}  &           5.59             &        -2.006        &     15.3            \\
          11125.58         &  \ion{Fe}{II}  &           5.62             &        -2.213        &     10.5            \\
          13251.14         &  \ion{Fe}{II}  &           9.41             &         0.768        &     13.4            \\
        \hline
  \end{longtable}
\end{onecolumn}

\end{appendix}

\end{document}
