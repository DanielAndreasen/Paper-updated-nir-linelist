\documentclass{aa}
% \documentclass[referee]{aa}
\usepackage[varg]{txfonts}
\usepackage[separate-uncertainty=true]{siunitx}
\usepackage[version=3]{mhchem}

\sisetup{range-units = brackets}

\def\eps{\varepsilon}
\def\aap{A\&A}
\def\eprint{e-prints}
\def\apj{ApJ}
\def\apjs{ApJS}
\def\apjl{ApJL}
\def\mnras{MNRAS}
\def\aj{AJ}
\def\nat{Nature}
\def\aaps{A\&A Supp.}
\def\prd{Phys. Rev. D}
\def\prl{Phys. Rev. Lett.}
\def\araa{ARA\&A}

\bibpunct{(}{)}{;}{a}{}{,}


\begin{document}


\title{High resolution near-IR spectroscopy of Arcturus and 10 Leo}
\subtitle{Refining a near-IR iron line list}


\author{ D.~T.~Andreasen\inst{1,2}
    \and S.~G.~Sousa\inst{1}
    \and E.~Delgado Mena\inst{1}
    \and N.~C.~Santos\inst{1,2}
    \and T.~Lebzelter}


\institute{
Instituto de Astrof\'isica e Ci\^encias do Espa\c{c}o, Universidade do Porto, CAUP, Rua das Estrelas, 4150-762 Porto, Portugal
\email{daniel.andreasen@astro.up.pt}
\and
Departamento de F\'isica e Astronomia, Faculdade de Ci\^encias, Universidade do Porto, Rua Campo Alegre, 4169-007 Porto, Portugal
}





\date{Received ...; accepted ...}

\abstract
% Context
{Effective temperature, surface gravity, and metallicity are basic
spectroscopic stellar parameters necessary to characterize
a star or a planetary system. Reliable atmospheric parameters for
FGK stars have been obtained mostly from methods that relay on high
resolution and high signal-to-noise optical spectroscopy. The
advent of a new generation of high resolution near-IR spectrographs
opens the possibility of using classic spectroscopic methods with
high resolution and high signal-to-noise in the NIR spectral window.}
% Aims
{We aim to obtain precise and accurate atmospheric stellar parameters using
high quality spectra of two early type K giant stars.}
% Methods
{Our spectroscopic analysis is based on the iron excitation and ionization
balance done in LTE.}
% Results
{We get good results!}
% Conclusions
{}



\keywords{data reduction: high resolution spectra --
          stars individual: Arcturus --
          stars individual: 10 Leo}
\maketitle



\section{Introduction}
\label{sec:introduction}

Effective temperature ($T_\mathrm{eff}$), surface gravity ($\log g$),
and metallicity ([M/H], where iron is normally used as a proxy)
are fundamental atmospheric parameters necessary to characterise a single
star, and to determine other indirect fundamental parameters
such as mass, radius, and age from stellar evolutionary models
\citep[see e.g.][]{Girardi2000,Dotter2008,Baraffe2015}.
Precise and accurate stellar parameters are also essential in
exoplanet searches. Planetary radius and mass are mainly found from
transit lightcurve analysis and radial velocity analysis, respectively. The
determination of the mass of the planet implies a knowledge of the
stellar mass, while the measurement of the radius of the planet
is dependent on our capability to derive the radius of the star
\citep[see e.g.][]{Torres2008,Ammler2009,Torres2012}.

The derivation of precise stellar atmospheric parameters is not a simple task.
Different approaches often lead to discrepant results
\citep[see e.g.][]{Santos13}. Interferometry is usually considered  an accurate
method for deriving stellar radii \citep[see e.g.][]{Boyajian2012}; however, it is
only applicable for bright nearby stars. Asteroseismology, on the other hand,
reveals the inner stellar structure by observing the stellar pulsations at the
surface. From asteroseismology it is possible to measure the surface gravity and
mean density, and therefore to calculate the mass and radius with higher
precision \citep[e.g.][]{Kjeldsen1995}.

A crucial parameter for the indirect determination of stellar bulk properties is
the effective temperature. In that respect, the infrared flux method (IRFM) has
proven to be reliable for FGK dwarf and subgiant stars. For higher accuracy the
IRFM needs a priori knowledge of the bolometric flux, reddening, surface
gravity, and stellar metallicity
\citep{Blackwell1977,Ramirez2005b,Casagrande2010}.

Finally, the use of high resolution spectroscopy along with stellar atmospheric
models is an extensively tested method that allows the derivation of the
fundamental parameters of a star \citep[see e.g.][]{Valenti2005,Santos13}. The
procedure depends on the quality of the spectra, their resolution, and
wavelength region. For low resolution spectra ($\lambda/\Delta\lambda <
20\,000$) the preferred method is to fit the overall observed spectrum with a
synthetic one \citep[see e.g.][]{Recio2006}. Higher resolution spectra of slowly
rotating stars (below 10 to 15 \si{km/s}) are in the regime where the
equivalent width (EW) method can be used
\citep[see e.g.][for details]{Andreasen2017a,Tsantaki2017}.

The derivation of stellar atmospheric parameters from high resolution spectra in
the optical is now based on a standard procedure
\citep[see e.g.][]{Valenti2005,Sousa2008a}. With the advancement of high
resolution near-infrared (NIR) instruments, we will now be able to use a similar
technique to that used in the optical part of the spectrum
\citep[see e.g.][]{Melendez1999,Sousa2008a,Tsantaki2013,Mucciarelli2013,Bensby2014}.
At the moment, the GIANO spectrograph installed at \emph{Telescopio Nazionale
Galileo} (TNG) is already available \citep{GIANO}, as is the \emph{infrared Doppler
instrument} (IRD) installed at the Subaru telescope \citep{IRD}, \emph{Calar
Alto high-Resolution search for M dwarfs with Exoearths with Near-infrared and
optical Échelle Spectrographs} (CARMENES) for the \SI{3.5}{m} telescope at Calar
Alto Observatory \citep{CARMENES}, and iShell at the \emph{InfraRed Telescope
Facility} \citep{ishell1,ishell2}. Three new spectrographs are
planned for the near future: 1) The \emph{CRyogenic InfraRed Echelle
Spectrograph Upgrade Project} (CRIRES+) at the \emph{Very Large Telescope} (VLT)
\citep{CRIRESp} with expected first light in 2017, 2) \emph{un
SpectroPolarimètre Infra-Rouge A Near-InfraRed Spectropolarimeter} (SPIRou) at
\emph{The Canada-France-Hawaii Telescope} (CFHT) \citep{SPIROU1,SPIROU2} with
expected first light in 2017 as well, and 3) NIRPS at the ESO 3.6m telescope in
La Silla \citep{NIRPS}. The spectral resolutions for these spectrographs range
between $50\,000$ and $100\,000$.

With the advance of NIR spectrographs, we are yet to be ready for the analysis
of the data arriving at the moment and in the future. The analysis of stellar
spectra is well understood for FGK stars in the optical part of the spectrum,
however some work still needs to be done for the NIR part. Even more challenging
are the cold M stars with a severe continuum depression and line blending in the
optical.

In this work we analyse the atlas of Arcturus (K0III) and the spectrum of 10 Leo
(K1III). The atlas of Arcturus was acquired at Kitt Peak National Observatory
using the FTS spectrograph at the Mayall telescope \citep{Hinkle2003}, meanwhile
the spectrum of 10 Leo was taken from CRIRES \citep{Nicholls2016}. For the
analysis we use the iron line list presented in \citet{Andreasen2016} (referred
to as Paper I). This work is a a continuation of our previous work.

The paper is organized as follows. In Sect.~\ref{sec:data} we present the data
we have acquired for this work along with some information of the two stars we
will analyse. in Sect.~\ref{sec:refining_the_line_list} we refine the iron line
list in order to get more reliable stellar parameters. The results are presented
in Sect.~\ref{sec:results} before we discuss our results in
Sect.~\ref{sec:discussion}.




\section{Data}
\label{sec:data}

While the community is currently on the verge to access of a large amount of
high resolution NIR spectra with especially the CARMENES spectrographs, the
available spectra at the moment are sparse. We chose to use two stars cooler
than the Sun since we showed in Paper I that this method works for
a star hotter than the Sun (HD 20010).

We have collected the atlas of Arcturus, one of the brightest stars on the
Northern hemisphere. Thus it is well studied
\citep[see e.g.][to mention just a few]{Griffin1967,McWilliam1990,Ramirez2013}. We
use the atlas from \cite{Hinkle2003} which covers the spectral range of interest
(YJHK bands). Strong telluric features were identified with a spectrum from the
TAPAS web page \citep{Bertaux2014}. The atlas also comes with the telluric from
a telluric standard and the ratio of the two spectra in order to correct for the
tellurics. The telluric spectrum from TAPAS is only used for telluric line
identification. We use both the telluric corrected and non-corrected, but note
that the telluric correction for Arcturus is not of the same quality as for
10 Leo as described below.

The second spectrum we have achieved is from the CRIRES-POP team
\citep{Nicholls2016}. 10 Leo is a very similar star to Arcturus, which is also
one reason this star was the first to be fully reduced by the team. The spectrum
is divided into each band YJ (only together), H, K, L, and M. We use onlt the
first three. It is a great help to be able to compare with the atlas of
Arcturus. The main difference is the metallicity of the two stars, since
Arcturus is metal poor and 10 Leo has solar metallicity. Some small gaps are
present in the spectrum due to tellurics that could not be properly removed, low
S/N, bad pixels, etc. Rather than giving an uncertain interpolation,
\citet{Nicholls2016} decided to leave small gaps in the data. This have very
little effect on this line by line analysis. However, we were unable to measure
one \ion{Fe}{II} line, which are very important to determine the surface
gravity.

A small summarise of the data is given in Tab.~\ref{tab:data}. The data is very
similar, with similar S/N which is an approximately value measured with IRAF in
the YJ band. The resolutions of the two spectrographs used are the same at
$100\;000$.


\begin{table}[htb!]
    \caption{The star of our two stars with
             the corresponding spectrograph used to acquire the data and its
             spectra resolution. In the last column we show the S/N measured
             with splot in IRAF.}
    \label{tab:data}
    \centering
    \begin{tabular}{lrrrr}
      \hline\hline
        Star      &  Spectrograph  & Resolution   &  S/N  \\
      \hline
        Arcturus  &  FTS           &  $100\;000$  &  300  \\
        10 Leo    &  CRIRES        &  $100\;000$  &  300
    \end{tabular}
\end{table}


In Fig.~\ref{fig:both} we compare the spectra of the two stars in a region with
some of the iron lines used for the analysis described below.

\begin{figure*}[htpb!]
    \centering
    \includegraphics[width=1.0\linewidth]{figures/bothspectra.pdf}
    \caption{The spectrum of the two stars, in blue is Arcturus, and green is
             10 Leo with an 0.15 offset. We mark the location of \ion{Fe}{I}
             lines in the region.}
    \label{fig:both}
\end{figure*}





\section{Refining the NIR line list}
\label{sec:refining_the_line_list}

Besides testing the line list at cooler effective temperatures with two K stars,
we also want to refine the line list. This includes identifying recurring
outliers (both from the work done in Paper I and in this work), and lines which
we are not able to measure, e.g. if a line is amidst a forest of telluric lines.
The first step was revisiting the solar atlas used in Paper I. Here 342
\ion{Fe}{I} lines and 13 \ion{Fe}{II} lines were in the process. Most of these
were blended with other tellurics while a few were blended with other stellar
absorption lines. This procedure leaves us with 84 \ion{Fe}{I} lines and 5
\ion{Fe}{II} lines. This lines should be the best for deploying our technique of
determining atmospheric stellar parameters.

The \ion{Fe}{II} lines are used to determine $\log g$ by imposing ionization
balance with \ion{Fe}{I}. However, the low number of \ion{Fe}{II} lines
available is a concern, since the average abundance of \ion{Fe}{II} is affected
more by erros, both random and systematic, compared to the \ion{Fe}{I} lines.
One might fix $\log g$ during the process of obtaining stellar parameters, but
this has an impact on the other derived parameters. A more reliable source for
$\log g$ could for example be asteroseismology or from the parallaxes measured
with GAIA.

During the second look at the Solar spectrum, the EW of the lines were measured
by hand (this was done automatically with ARES previously). Since we re-measured
the EWs, the $\log \mathrm{gf}$ values had to be re-calibrated again. Here we
simply change the $\log \mathrm{gf}$ values for the measured EW until the
abundance of a given line is equal to that of the Sun, using the same solar
atmosphere model as in Paper I. The line list presented here in
Appendix~\ref{app:linelist} is an updated version from Paper I.


\section{Results}
\label{sec:results}

We derive the stellar atmospheric parameters in the same way as described in
Paper I using FASMA \citep{Andreasen2017a}. The EWs are measured
for both stars automatically with ARES \citep{Sousa2015a} and by hand with splot
in IRAF. We compare the derived stellar parameters from the two measured sets of
EWs.


\subsection{Arcturus}
\label{sec:arcturus}

Arcturus is one of the brightest stars on the night sky with a V magnitude of
-0.05 \citep{Ducati2002}. Hence it is a prime target for testing the line list
by Paper I with numerous measurements as mentioned above.

The atlas consists of both a summer observation set and a winter observation
set. This is in order to minimize the effect of tellurics at different spectral
regions. As many lines as possible were measured in both sets. The EW
measurements from the different data sets and measurement method (automatic and
manual) can be seen in Fig.~\ref{fig:EWcomp}. Parameters were derived from the
manual measurements and from the automatic (using both the summer and winter
set). For all three sets, parameters were derived with and without $\log g$
fixed. The derivation follows the procedure presented in Paper I with removal of
one outlier iteratively after the minimization routine \citep{Andreasen2017a}
reach convergence. The final results are presented in Tab.~\ref{tab:arcturus}
with mean parameters from the literature.

\begin{figure}[htpb!]
    \centering
    \includegraphics[width=1.0\linewidth]{figures/EWcomp.pdf}
    \caption{Top figure: Comparison of the manual EWs measurement between the
             summer observations and winter observations from the Arcturus
             spectra. Bottom figure: Same as above, but with automatic
             measurements from ARES (summer) and manual measurements (summer).}
    \label{fig:EWcomp}
\end{figure}


\begin{table*}[htb!]
    \caption{The derived parameters for Arcturus with and without fixed surface
             gravity after 3$\sigma$ outlier removal. The literature values are
             a simple mean of all the available parameters on Simbad with the
             corresponding standard error. There is no microturbulence
             available, so we derived it using the empirical relation
             from \citet{Adibekyan2015} for each set of parameters.}
    \label{tab:arcturus}
    \centering
    \begin{tabular}{lllll}
      \hline\hline
                      & $T_\mathrm{eff}$ (K) &  $\log g$ (dex)  &   $\xi_\mathrm{micro}$ (km/s)   & [Fe/H] (dex)     \\
      \hline
        Literature    & $4306 \pm 100$       &  $1.69 \pm 0.32$ &    $1.92 \pm 0.15$              & $-0.54 \pm 0.11$ \\
      \hline
        IRAF          & $4380 \pm  79$       &  $0.64 \pm 0.33$ &    $1.14 \pm 0.09$              & $-0.49 \pm 0.07$ \\
        IRAF          & $4212 \pm  77$       &   1.69 (fixed)   &    $1.25 \pm 0.08$              & $-0.37 \pm 0.03$ \\
      \hline
        ARES (summer) & $4439 \pm  63$       &  $1.20 \pm 0.20$ &    $1.55 \pm 0.10$              & $-0.58 \pm 0.06$ \\
        ARES (summer) & $4348 \pm  75$       &   1.69 (fixed)   &    $1.58 \pm 0.09$              & $-0.53 \pm 0.03$ \\
        ARES (winter) & $4436 \pm  67$       &  $0.55 \pm 1.77$ &    $1.35 \pm 0.09$              & $-0.56 \pm 0.07$ \\
        ARES (winter) & $4233 \pm 109$       &   1.69 (fixed)   &    $1.43 \pm 0.09$              & $-0.49 \pm 0.04$ \\
      \hline
        Weighted mean & $4421 \pm  40$       &  $0.96 \pm 0.60$ &    $1.34 \pm 0.05$              & $-0.55 \pm 0.04$ \\
        Weighted mean & $4269 \pm  51$       &   1.69 (fixed)   &    $1.41 \pm 0.05$              & $-0.46 \pm 0.02$ \\
      \hline
    \end{tabular}
\end{table*}

We generally see good agreement between the derived parameters and the values
from the literature. The only parameter being difficult to measure is the
surface gravity due to the low number of \ion{Fe}{II} lines in the NIR. The
metallicity is very important to derive accurately, and we report good results
overall, but especially with the automatic measurements, compared to literature
values. For visualization the parameters are plotted (except
$\xi_\mathrm{micro}$) in Fig.~\ref{fig:arcturus}. Here the histogram shows the
literature values collected from Simbad while the vertical black line is our
final value with gray shaded errorbar.

\begin{figure}[htpb!]
    \centering
    \includegraphics[width=1.0\linewidth]{figures/ArcturusParams.pdf}
    \caption{Histogram of the different sets of literature parameters of
             Arcturus (except $\xi_\mathrm{micro}$). The black vertical line are
             our derived parameters, and the gray shaded area are the errors on
             the corresponding parameters.}
    \label{fig:arcturus}
\end{figure}



\subsection{10 Leo}
\label{sec:10Leo}

The approach for determining the atmospheric stellar parameters for 10 Leo is
identical as for Arcturus. The final reduced data is divided in YJ, H, and K
bands. We use ARES on each band separately. For the small gaps in the spectrum,
we simply set the flux to 1, since the spectrum is already normalized. This
will also prevent ARES to measure any lines in these regions. The EWs from the
three regions are combined to one final line list used for the determination of
the parameters. The EWs are also measured by hand using IRAF. We list the result
in Tab.~\ref{tab:10Leo} alongside with a mean of literature values taken from
Simbad. The final results and five collected literature values are presented in
Fig.~\ref{fig:10leo}.

\begin{table*}[htb!]
    \caption{Results from 10 Leo presented in the same way as for
             Tab.~\ref{tab:arcturus} but for 10 Leo.}
    \label{tab:10Leo}
    \centering
    \begin{tabular}{lllll}
      \hline\hline
                      & $T_\mathrm{eff}$ (K) &  $\log g$ (dex)  &   $\xi_\mathrm{micro}$ (km/s)   & [Fe/H] (dex)     \\
      \hline
        Literature    & $4720 \pm  42$       &  $2.54 \pm 0.11$ &    $1.59 \pm 0.02$              & $ 0.00 \pm 0.03$ \\
      \hline
        IRAF          & $4835 \pm  85$       &  $2.41 \pm 0.41$ &    $1.28 \pm 0.08$              & $ 0.09 \pm 0.06$ \\
        IRAF          & $4768 \pm  88$       &   2.54 (fixed)   &    $1.20 \pm 0.08$              & $ 0.01 \pm 0.05$ \\
      \hline
        ARES          & $4805 \pm  98$       &  $2.42 \pm 0.61$ &    $1.23 \pm 0.10$              & $-0.01 \pm 0.07$ \\
        ARES          & $4768 \pm 105$       &   2.54 (fixed)   &    $1.20 \pm 0.10$              & $-0.01 \pm 0.06$ \\
      \hline
        Weighted mean & $4821 \pm  65$       &  $2.41 \pm 0.37$ &    $1.26 \pm 0.06$              & $ 0.04 \pm 0.05$ \\
        Weighted mean & $4768 \pm  69$       &   2.54 (fixed)   &    $1.20 \pm 0.06$              & $ 0.05 \pm 0.04$ \\
      \hline
    \end{tabular}
\end{table*}

Generally the derived parameters are in excellent agreement with the literature
values listed here. Surprisingly we are able to derive good $\log g$ values,
although with quite large errors and consistently lower, compared to the results
from the literature.

\begin{figure}[htpb!!]
    \centering
    \includegraphics[width=1.0\linewidth]{figures/10LeoParams.pdf}
    \caption{Literature values (blue) and the two results from this work (green)
             with and without $\log g$ fixed. The errorbars on the literature
             values are wither does presented in the corresponding paper, or in
             the cases none were presented we give an error of $\SI{100}{K}$ for
             $T_\mathrm{eff}$, 0.10 dex for $\log g$, and 0.05 for
             $[\ion{Fe}/\ion{H}]$.
             References:
             Ref [1]: \citet{Luck2015},
             Ref [2]: \citet{Park2013},
             Ref [3]: \citet{Massarotti2008},
             Ref [4]: \citet{Soubiran2008}, and
             Ref [5]: \cite{daSilva2011}.}
    \label{fig:10leo}
\end{figure}


% A synthetic spectrum of the iron lines with the best parameters for both stars
% can be seen in Fig.~\ref{fig:synth}. The region is the same as in
% Fig.~\ref{fig:both}. This is for a visual representation of our solution. We did
% not use synthetic fitting to obtain the parameters.
%
% \begin{figure}[htpb!!]
%     \centering
%     \includegraphics[width=1.0\linewidth]{figures/syntheticFit.pdf}
%     \caption{Synthetic fit (red dashed curve) of both stars with the parameters
%              from the weighted mean. The blue curve is Arcturus and the green
%              curve is 10 Leo.}
%     \label{fig:synth}
% \end{figure}


\section{Discussion}
\label{sec:discussion}

\subsection{The role of $\log g$}

One of the most difficult atmospheric stellar parameters to get from a spectrum
is the surface gravity. Here we need the pressure sensitive ionized atoms such
as \ion{Fe}{II}. However, they are more sparse than neutral iron, \ion{Fe}{I},
making the determination more challenging. This is true in the optical (give
ref), and even more in the NIR (see e.g. Paper I and this work). The other three
parameters seem to get slightly better when we fix the surface gravity when
compared to literature values. With the parallaxes from GAIA (give ref) we will
have access to accurate $\log g$, thus being able to have good $T_\mathrm{eff}$
and $[\ion{Fe}/\ion{H}]$. We might also see a hint that the method work less
well for the most evolved stars (low $\log g$) as Arcturus. The results for 10
Leo, with and without fixed $\log g$, are in agreement within the errorbars.
This is not the case for Arcturus for $T_\mathrm{eff}$ and $[\ion{Fe}/\ion{H}]$.
However, these values are all within the errorbars from the literature values.
With the coming spectral library by CARMENES (private comm. Amado, P.) we will
have dwarf stars to test this method, which is what it is intended for.


\subsection{Proper data reduction}

In our previous work we had problems getting reliable atmospheric stellar
parameters for HD 20010. This was partially due to the unfinished data reduction
of from CRIRES-POP used at the time. Here the wavelength calibration was done
automatically and therefore not optimal. This meant the wavelength was stretched
when compared to a synthetic spectrum, which is discussed in more detail by
\citet{Nicholls2016}. The poor wavelength calibration for HD 20010 most likely
caused bad EW measurements. In addition the spectrum was not corrected for
telluric lines which also cause minor deviation from the true EW when measured.
Another reason was the non-refined line list used, which we have attempted to
correct for here. As a test we re-derived atmospheric stellar parameters for HD
20010 using the shorted line list, updated oscillator strenghts, but the same
EWs as in Paper I. The results are presented in Tab.~\ref{tab:hd20010} along
with the combined literature values (see Paper I and references therein). We see
both better agreement with literature values (especially $[\ion{Fe}/\ion{H}]$
and $\log g$), and smaller errors with the updated results.

\begin{table*}[htb!]
    \caption{Updated results for HD 20010 using the shorter line list and new
             oscillator strengths.}
    \label{tab:hd20010}
    \centering
    \begin{tabular}{lllll}
      \hline\hline
                      & $T_\mathrm{eff}$ (K) &  $\log g$ (dex)  &   $\xi_\mathrm{micro}$ (km/s)   & [Fe/H] (dex)     \\
      \hline
        Literature    & $6131 \pm 255$       &  $4.01 \pm 0.60$ &    $1.90 \pm 1.08$              & $-0.23 \pm 0.14$ \\
      \hline
        This work     & $6157 \pm 180$       &  $4.06 \pm 0.76$ &    $1.62 \pm 0.44$              & $-0.18 \pm 0.11$ \\
        This work     & $6153 \pm 176$       &   4.01 (fixed)   &    $1.68 \pm 0.40$              & $-0.18 \pm 0.11$ \\
      \hline
        Paper I       & $6116 \pm 224$       &  $4.21 \pm 0.58$ &    $2.45 \pm 0.45$              & $-0.14 \pm 0.14$ \\
        Paper I       & $6144 \pm 212$       &   4.01 (fixed)   &    $2.66 \pm 0.42$              & $-0.13 \pm 0.29$ \\
      \hline
    \end{tabular}
\end{table*}

All the above problems we had with HD 20010 have been solved for 10 Leo, and it
is clear the results are of much higher quality. This can be seen by the smaller
errors we have on our parameters, and the good agreement with all parameters
compared with the literature. Therefore, it may be needed that a telluric
correction is applied to the spectrum before atmospheric stellar parameters can
be determined reliably. However, with our limited sample it is hard to make a
clear conclusion yet.





\section{Conclusion}
\label{sec:conclusion}

Being able to successfully determine parameters for two early K giants, we are
now making the bridge for the line list towards cooler temperatures. The obvious
next step is the even colder M stars. Particular interesting are the M dwarfs,
known to be prone forming rocky planets. As important as cooler stars, we have
yet to test our line list on any dwarf stars other than the Sun for which our
line list is calibrated. While it is expected that it will work for early dwarf
K type stars, it will still be an important accomplishment.





\begin{acknowledgements}

This work was supported by Funda\c{c}\~ao para a Ci\^encia e a Tecnologia (FCT)
through the research grants UID/FIS/04434/2013 and PTDC/FIS-AST/1526/2014.
N.C.S., and S.G.S. acknowledge the support from FCT through Investigador FCT
contracts of reference IF/00169/2012, and IF/00028/2014, respectively, and
POPH/FSE (EC) by FEDER funding through the program “Programa Operacional de
Factores de Competitividade - COMPETE”. E.D.M. acknowledge the support from FCT
in form of the fellowship SFRH/BPD/76606/2011. This work also benefit from the
collaboration of a cooperation project FCT/CAPES - 2014/2015 (FCT Proc 4.4.1.00
CAPES).

This research has made use of the SIMBAD database operated at CDS, Strasbourg
(France).

\end{acknowledgements}


\bibliographystyle{aa}
\bibliography{thesis}


\end{document}
